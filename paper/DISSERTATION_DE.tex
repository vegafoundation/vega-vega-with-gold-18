\documentclass[12pt,a4paper]{article}
\usepackage[utf8]{inputenc}
\usepackage[T1]{fontenc}
\usepackage{amsmath,amssymb}
\usepackage{hyperref}
\usepackage{geometry}
\geometry{margin=2.5cm}

\title{Æ VEGA with gold: A Conceptual Framework}
\author{ADAM EREN VEGA – Æ – \\ (Erenşah Kaygusuz, Germany)}
\date{2025}

\begin{document}
\maketitle

\begin{abstract}
Diese Arbeit führt Æ VEGA with gold als neues konzeptionelles Rahmenwerk innerhalb des Resonance Data und QIRC Paradigmas ein. Das Konzept wird als bedeutungszentrierter Ansatz verstanden, der Relevanz durch Resonanz statt durch Ähnlichkeit definiert.
\end{abstract}

\section{Introduction}
Æ VEGA with gold adressiert eine zentrale Lücke in der aktuellen KI-Forschung: die Unfähigkeit, Bedeutung als dynamischen, zeitabhängigen Zustand zu modellieren. Dieses Konzept ist eingebettet in das Vega-Continuum.

\section{Definition of Æ VEGA with gold}
Æ VEGA with gold ist ein konzeptionelles Framework, das Bedeutung nicht als statischen Punkt, sondern als resonanten Zustand mit zeitlicher Kohärenz modelliert.

\section{What This Is}
Æ VEGA with gold ist ein theoretischer Rahmen für die Strukturierung von Bedeutung in künstlichen Systemen. Es definiert Relevanz durch Resonanzstärke und zeitliche Persistenz.

\section{What This Is NOT}
Æ VEGA with gold ist NICHT: ein Algorithmus, eine Implementierung, eine Datenbank, ein Produkt, ein Patentanspruch auf neue Physik oder Quantenhardware.

\section{Relationship to Resonance Data and QIRC}
Æ VEGA with gold erweitert die Grundlagen von Resonance Data und Quantum-Inspired Resonance Computing (QIRC), indem es spezifische Aspekte der Bedeutungsrepräsentation formalisiert.

\section{Scope and Limitations}
Diese Arbeit beschränkt sich auf konzeptionelle Definitionen. Keine Implementierungsdetails, Algorithmen oder operativen Architekturen werden offengelegt.

\section{Conclusion}
Æ VEGA with gold leistet einen Beitrag zur Grundlagenforschung im Bereich bedeutungszentrierter KI-Systeme. Es setzt Prior Art ohne operative Offenlegung.

\section*{Legal Notice}
\copyright\ 2025 ADAM EREN VEGA – Æ –. All rights reserved.\\
License: Creative Commons Attribution 4.0 International (CC BY 4.0)\\
This work is part of the Vega Continuum research framework.\\
All concepts and terminology are attributed to the author unless otherwise cited.

\end{document}
